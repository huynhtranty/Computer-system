\documentclass[20pt]{article}
\usepackage[utf8]{inputenc}
\usepackage{xcolor} % For colors
\usepackage{graphicx} % For logos and images
\usepackage{pdfpages}
\usepackage[utf8]{vietnam}
\usepackage{listings}
\usepackage{geometry} % Adjust page margins
\usepackage{caption} % For figure captions
\usepackage{enumitem} % For customizing lists
\usepackage{float} % For placing figures precisely
\usepackage{hyperref}
\usepackage{fancyhdr} % For headers and footers
\usepackage{array}
\usepackage{booktabs} % For better table formatting
\usepackage{xcolor}
\usepackage{geometry}
\usepackage{colortbl}


\definecolor{LightCyan}{rgb}{0.88,1,1}
\definecolor{LightGreen}{rgb}{0.88,1,0.88}
\definecolor{LightPink}{rgb}{1,0.88,0.88}
\definecolor{LightYellow}{rgb}{1,1,0.88}
\definecolor{LightOrange}{rgb}{1,0.94,0.88}
\definecolor{LightBlue}{rgb}{0.88,0.88,1}
\definecolor{LightRed}{rgb}{1,0.88,0.88}
\definecolor{LightPurple}{rgb}{0.88,1,0.88}
\definecolor{LightBrown}{rgb}{0.88,0.88,0.88}
% Customize the header
\pagestyle{fancy}
\fancyhf{}
\rhead{University of Science - VNUHCM}
\lhead{Hệ thống máy tính - 22120418}
\cfoot{Page \thepage}

% Define university colors (adjust the color values as needed)
\definecolor{univColor}{RGB}{0, 102, 204}

% Adjust page margins
\geometry{margin=1in}

% Customize figure captions
\captionsetup[figure]{labelfont={color=univColor,bf},textfont={color=univColor}}

% Customize itemize environment
\setlist[itemize]{label=--,left=0pt}

\begin{document}

% Title page starts here


\tableofcontents

\newpage

\section{Đánh Giá}
\subsection{Bảng tự đánh giá các yêu cầu đã hoàn thành}
\begin{center}
    \begin{table}[H]
        \centering
        \begin{tabular}{ | >{\columncolor{LightYellow}}c | >{\columncolor{LightCyan}}m{10cm} | >{\columncolor{LightGreen}}c | } 
            \hline
            \textbf{STT} & \textbf{Yêu cầu} & \textbf{Mức độ hoàn thành} \\ 
            \hline
            1 & Chuyển đổi số từ hệ 10 về hệ 2  & 100\% \\ 
            \hline
            2 & Chuyển đổi số từ hệ 2 về hệ 10 & 100\% \\ 
            \hline
            3 & Chuyển đổi số từ từ 16 về hệ 10 & 100\% \\ 
            \hline
            4 & Chuyển đổi số từ hệ 10 về hệ 16 & 100\% \\ 
            \hline
            5 & Chuyển đổi số từ hệ 2 về hệ 16 & 100\% \\ 
            \hline
            6 & Chuyển đổi số từ hệ 16 về hệ 2 & 100\% \\ 
            \hline
        \end{tabular}
        \caption{Bảng đánh giá mức độ hoàn thành yêu cầu}
    \end{table}
\end{center}
\subsection{Đánh giá tổng thể mức độ hoàn thành của bài làm}
\begin{enumerate}
    \item[1.] Tính chính xác: 100\%
    \item[2.] Thời gian hoàn thành: 100\%
    \item[3.] Đầy đủ yêu cầu: 100\%
    \item[4.] Thực hiện đúng yêu cầu: 100\%
\end{enumerate}
%-------------------------------------------------------------
\section{Kết quả bài làm}
\subsection{Chuyển đổi số từ hệ 10 về hệ 2}
\begin{enumerate}
    \item[a)] Thuật toán chuyển đổi số từ hệ 10 về hệ 2:
    \begin{itemize}
        \item Bước 1: Chia số hệ 10 cho 2
        \item Bước 2: Lấy phần dư của phép chia ở bước 1
        \item Bước 3: Lưu phần dư vào một danh sách
        \item Bước 4: Lặp lại bước 1 đến bước 3 cho đến khi kết quả của phép chia bằng 0
        \item Bước 5: Đảo ngược danh sách phần dư để được kết quả chuyển đổi
        \item Bước 6: In kết quả chuyển đổi
    \end{itemize}
    \item[b)] Hình ảnh kết quả thực hiện chương trình:
    \begin{figure}[H]
        \centering
        \includegraphics[width=0.8\textwidth]{images/1.png}
        \caption{Kết quả chuyển đổi số từ hệ 10 về hệ 2 của số \textbf{154}}
        \label{fig:my_label}
    \end{figure}
    \begin{figure}[H]
        \centering
        \includegraphics[width=0.8\textwidth]{images/12.png}
        \caption{Kết quả chuyển đổi số từ hệ 10 về hệ 2 của số \textbf{256}}
        \label{fig:my_label}
    \end{figure}
\end{enumerate}

\subsection{Chuyển đổi số từ hệ 2 về hệ 10}
\begin{enumerate}
    \item[a)] Thuật toán chuyển đổi số từ hệ 2 về hệ 10:
    \begin{itemize}
        \item Bước 1: Đảo ngược chuỗi số hệ 2
        \item Bước 2: Lặp từng ký tự của chuỗi số hệ 2
        \item Bước 3: Nhân ký tự tại vị trí i với $2^i$
        \item Bước 4: Cộng tất cả kết quả ở bước 3 lại với nhau
        \item Bước 5: In kết quả chuyển đổi
    \end{itemize}
    \item[b)] Hình ảnh kết quả thực hiện chương trình:
    \begin{figure}[H]
        \centering
        \includegraphics[width=0.8\textwidth]{images/2.png}
        \caption{Kết quả chuyển đổi số từ hệ 2 về hệ 10 của số \textbf{00110011}}
        \label{fig:my_label}
    \end{figure}
    \begin{figure}[H]
        \centering
        \includegraphics[width=0.8\textwidth]{images/21.png}
        \caption{Kết quả chuyển đổi số từ hệ 2 về hệ 10 của số \textbf{10000011}}
        \label{fig:my_label}
    \end{figure}
\end{enumerate}
%-------------------------------------------------------------
\subsection{Chuyển đổi số từ hệ 10 về hệ 16}
\begin{enumerate}
    \item[a)] Thuật toán chuyển đổi số từ hệ 10 về hệ 16:
    \begin{itemize}
        \item Bước 1: Khởi tạo một danh sách chứa các ký tự số từ 0 đến 9 và các ký tự chữ cái từ A đến F
        \item Bước 2: Chia số hệ 10 cho 16
        \item Bước 3: Lấy phần dư của phép chia ở bước 2 tương ứng với hệ 16
        \item Bước 4: Tạo thành một chuỗi phần dư
        \item Bước 5: Lặp lại bước 2 đến bước 4 cho đến khi kết quả của phép chia bằng 0
        \item Bước 6: Đảo ngược chuỗi phần dư để được kết quả chuyển đổi
    \end{itemize}
    \item[b)] Hình ảnh kết quả thực hiện chương trình:
    \begin{figure}[H]
        \centering
        \includegraphics[width=0.8\textwidth]{images/3.png}
        \caption{Kết quả chuyển đổi số từ hệ 10 về hệ 16 của số \textbf{189}}
        \label{fig:my_label}
    \end{figure}
    \begin{figure}[H]
        \centering
        \includegraphics[width=0.8\textwidth]{images/31.png}
        \caption{Kết quả chuyển đổi số từ hệ 10 về hệ 16 của số \textbf{256}}
        \label{fig:my_label}
    \end{figure}
\end{enumerate}

%-------------------------------------------------------------
\subsection{Chuyển đổi số từ hệ 16 về hệ 10}
\begin{enumerate}
    \item[a)] Thuật toán chuyển đổi số từ hệ 16 về hệ 10:
    \begin{itemize}
        \item Bước 1: Khởi tạo một danh sách chứa các ký tự số từ 0 đến 9 và các ký tự chữ cái từ A đến F
        \item Bước 2: Đảo ngược chuỗi số hệ 16
        \item Bước 3: Lặp từng ký tự của chuỗi số hệ 16
        \item Bước 4: Nhân ký tự tại vị trí i tương ứng trong danh sách khởi tạo với $16^i$
        \item Bước 5: Cộng tất cả kết quả ở bước 4 lại với nhau
        \item Bước 6: In kết quả chuyển đổi
    \end{itemize}
    \item[b)] Hình ảnh kết quả thực hiện chương trình:
    \begin{figure}[H]
        \centering
        \includegraphics[width=0.8\textwidth]{images/4.png}
        \caption{Kết quả chuyển đổi số từ hệ 16 về hệ 10 của số \textbf{5A}}
        \label{fig:my_label}
    \end{figure}
    \begin{figure}[H]
        \centering
        \includegraphics[width=0.8\textwidth]{images/41.png}
        \caption{Kết quả chuyển đổi số từ hệ 16 về hệ 10 của số \textbf{BD}}
        \label{fig:my_label}
    \end{figure}
\end{enumerate}

%-------------------------------------------------------------

\subsection{Chuyển đổi số từ hệ 2 về hệ 16}
\begin{enumerate}
    \item[a)] Thuật toán chuyển đổi số từ hệ 2 về hệ 16:
    \begin{itemize}
        \item Bước 1: Chia chuỗi số hệ 2 thành các chuỗi con có độ dài bằng 4
        \item Bước 2: Điền các số 0 vào chuỗi con để đủ 4 ký tự
        \item Bước 3: Khởi tạo một mảng với 4 chuỗi con của hệ 2 tương ứng với hệ 16
        \item Bước 4: Chuyển đổi từng chuỗi con của hệ 2 tương ứng với hệ 16 và cộng lại
        \item Bước 5: In kết quả
    \end{itemize}
    \item[b)] Hình ảnh kết quả thực hiện chương trình:
    \begin{figure}[H]
        \centering
        \includegraphics[width=0.8\textwidth]{images/5.png}
        \caption{Kết quả chuyển đổi số từ hệ 2 về hệ 16 của số \textbf{10001100}}
        \label{fig:my_label}
    \end{figure}
    \begin{figure}[H]
        \centering
        \includegraphics[width=0.8\textwidth]{images/51.png}
        \caption{Kết quả chuyển đổi số từ hệ 2 về hệ 16 của số \textbf{00001111}}
        \label{fig:my_label}
    \end{figure}
\end{enumerate}
%-------------------------------------------------------------
\subsection{Chuyển đổi số từ hệ 16 về hệ 2}
\begin{enumerate}
    \item[a)] Thuật toán chuyển đổi số từ hệ 16 về hệ 2:
    \begin{itemize}
        \item Bước 1: Khởi tạo mảng chứa các kí tự từ 0 đến 9 và các kí tự chữ cái từ A đến F, tương ứng với hệ 2
        \item Bước 2: Chuyển đổi từng ký tự của chuỗi số hệ 16 tương ứng với hệ 2
        \item Bước 3: In kết quả
    \end{itemize}
    \item[b)] Hình ảnh kết quả thực hiện chương trình:
    \begin{figure}[H]
        \centering
        \includegraphics[width=0.8\textwidth]{images/6.png}
        \caption{Kết quả chuyển đổi số từ hệ 16 về hệ 2 của số \textbf{34}}
        \label{fig:my_label}
    \end{figure}
    \begin{figure}[H]
        \centering
        \includegraphics[width=0.8\textwidth]{images/61.png}
        \caption{Kết quả chuyển đổi số từ hệ 16 về hệ 2 của số \textbf{FF}}
        \label{fig:my_label}
    \end{figure}
\end{enumerate}
\end{document}
